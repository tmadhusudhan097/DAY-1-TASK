Objects, in JavaScript, is it’s most important data-type and forms the building blocks for modern JavaScript. These objects are quite different from JavaScript’s primitive data-types(Number, String, Boolean, null, undefined and symbol) in the sense that while these primitive data-types all store a single value each (depending on their types).

Objects are more complex and each object may contain any combination of these primitive data-types as well as reference data-types.

For Eg. If your object is a student, it will have properties like name, age, address, id, etc and methods like updateAddress, updateNam, etc.

Objects and properties
A JavaScript object has properties associated with it. A property of an object can be explained as a variable that is attached to the object. Object properties are basically the same as ordinary JavaScript variables, except for the attachment to objects.  You access the properties of an object with a simple dot-notation:

objectName.propertyName

var myCar = new Object();
myCar.make = 'Ford';
myCar.model = 'Mustang';
myCar.year = 1969;
Unassigned properties of an object are undefined (and not null).
myCar.color; // undefined

Properties of JavaScript objects can also be accessed or set using a bracket notation .
myCar['make'] = 'Ford';
myCar['model'] = 'Mustang';
myCar['year'] = 1969;
An object property name can be any valid JavaScript string, or anything that can be converted to a string, including the empty string. 

// four variables are created and assigned in a single go, 
// separated by commas
var myObj = new Object(),
    str = 'myString',
    rand = Math.random(),
    obj = new Object();
myObj.type              = 'Dot syntax';
myObj['date created']   = 'String with space';
myObj[str]              = 'String value';
myObj[rand]             = 'Random Number';
myObj[obj]              = 'Object';
myObj['']               = 'Even an empty string';console.log(myObj);
You can also access properties by using a string value that is stored in a variable:

var propertyName = 'make';
myCar[propertyName] = 'Ford';propertyName = 'model';
myCar[propertyName] = 'Mustang';


Creating Objects In JavaScript :
Create JavaScript Object with Object Literal
One of easiest way to create a javascript object is object literal, simply define the property and values inside curly braces as shown below
let bike = {name: 'SuperSport', maker:'Ducati', engine:'937cc'};


Creating JavaScript Object with Constructor:
function Vehicle(name, maker) {
   this.name = name;
   this.maker = maker;
}
let car1 = new Vehicle(’Fiesta’, 'Ford’);
let car2 = new Vehicle(’Santa Fe’, 'Hyundai’)
console.log(car1.name);    //Output: Fiesta
console.log(car2.name);    //Output: Santa Fe

Using the JavaScript Keyword new
The following example also creates a new JavaScript object with four properties:

Example
var person = new Object();
person.firstName = “John”;
person.lastName = “Doe”;
person.age = 50;
person.eyeColor = “blue”;


Using the Object.create method
Objects can also be created using the Object.create() method. T
// Animal properties and method encapsulation
var Animal = {
  type: 'Invertebrates', // Default value of properties
  displayType: function() {  // Method which will display type of Animal
    console.log(this.type);
  }
};
// Create new animal type called animal1 
var animal1 = Object.create(Animal);
animal1.displayType(); // Output:Invertebrates
// Create new animal type called Fishes
var fish = Object.create(Animal);
fish.type = 'Fishes';
fish.displayType(); 
// Output:Fishes
